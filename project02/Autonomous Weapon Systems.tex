Evan Chessman
Professor Armstrong
Senior Sem 401
March 3, 2015
 
Autonomous Weapon Systems
In this day and age we are in constant contact with increasingly autonomous technologies. With this expanding technology, we have a responsibility to consider the possible ramifications and come up with regulations and policies that control their use, especially in the weaponization of autonomous technologies. The Department of Defense has defined autonomous weapon systems as “ a weapon system that, once activated, can select and engage targets without further intervention by a human operator”(Schmitt, Michael N.). The Department of Defense goes on to express that human-supervised weapons systems have a design mechanism that allows the human operator to override the autonomous process of the weapon system, however the weapon system can select and enlist destinations and targets without the need for human input after the system has been activated. There are groups and organizations throughout the world that believe there should be a ban on this type of technology. The United Nations has met and come to various conclusions to the concerns that many of these groups and organizations are troubled about. This paper will focus on various topics in regards to autonomous weapon systems ranging from the varying definitions of autonomous systems, variables that comprise these systems, and the conflicting views of the people who are against and for autonomous weapons systems.
Autonomous weapons systems are complex structures that encompass a wide variety of functions. The United Nations Institute for Disarmament Research in an article has stated that “autonomy is a characteristic of a technology, attached to a function or functions, not an object in itself”(1, No). They also iterate that not all autonomous technologies have equal concern and in some instances and environments these autonomous technologies are beneficial. To help focus the discussion on regulating and creating policy on the international level, the UNIDIR has suggested creating a list of variables that target the most concerns. Such concerns may include goal satisfying actions, predictability, communication, depth of reasoning, precision of sensors and capacity for synthesis, bounds on location, and functions. (1, No)
Communication seems to be a large obstacle. The current concern with communication and autonomous weapon systems is; how much would the system have to interpret? Computers often need specific instructions and not in a way in which we speak. Also, how often would the need for communication be with the autonomous system? (1, No)
Environment seems to be one of the largest key concerns. Many of the above mentioned variables are interconnected with environment. The precision of sensors and capacity for synthesis often depend on the environment. In the article Framing Discussions of the Weaponization of Increasingly Autonomous Technologies it states that the “raw sensory capabilities of a system will determine its ability to discriminate things in the environment”. With a simple environment such as a field or a desert, there would not be a need for a high depth of reasoning. Unlike in an urban environment where the autonomous system would need a very high depth of reasoning for operation. The UNIDIR believes that with a simple environment the more predictable a system would be due to the fact that it would not need as many functions to work. Ultimately the environment type and conditions could play a major role in how much control over an autonomous weapon system we have. (1, No)
In May of 2014, a Convention on Certain Conventional Weapons was held. The four day meeting centered around autonomous weapon systems including factors such as sociological concerns, ethical use and military implementations of such systems. The UN News Center in an article stated that the four day meeting had record attendance. At the time, Acting Director-General of the United Nations Office at Geneva, Michael Moller, began the meeting by speaking to the assembled stating “you have the opportunity to take preemptive action and ensure that the ultimate decision to end life remains firmly under human control”(UN Meeting Targets 'killer Robots'). The UN Secretary-General questioned whether it was moral to leave the decision to use lethal force to autonomous weapon systems. Also, he questioned who would be held legally responsible if a war crime or other human rights violation occurred. The report of the 2014 Informal Meeting of Experts on Lethal Autonomous Weapons Systems mentioned the following aspects. (UN Meeting Targets 'killer Robots)
The session on technical aspects discussed “different levels of autonomy depending on the degree of human control”(Report of the 2014). It also brought up questions on the environment in which the system could operate in. Autonomy itself, should be measurable based off of various variables that were mentioned above by the UNIDIR. Also, discussions did bring up the fact that full autonomy had not been achieved.(Report of the 2014)
A session on ethical and sociological aspects discussed whether autonomous technologies could respond to moral dilemmas and that the abilities of these systems to acquire “moral reasoning and judgement was highly questionable” (Report of the 2014). The session also discussed whether or not values, ethics and “common elements to be incorporated into software” should be questioned. The main ethical question discussed was whether or not these systems should have the decision to take human life. (Report of the 2014)
There were two sessions on the legal aspects that discussed topics on accountability. Questions on who would be responsible if a crime took place looked at whether the programmers or manufactures could be held responsible in such instances. Topics related to human rights was also brought to question such as “right to life, human dignity, the right to be protected against inhuman treatment and the right to a fair trial”(Report of the 2014).
The military session discussed the impact autonomous weapon systems would have on peace and international law. The possibility of cyber attacks and operation environment were discussed as vulnerabilities. Some attendees had reservations on whether or not we could keep operation control over these systems, while some “delegations” claimed that they had no plans on even developing these technologies (Report of the 2014). The United States Navy has developed autonomous boats with no human control. However the decision for these watercraft to fire upon enemies is still in human control.(Hsu, Jeremy) 
There had not been any decision on whether or not to ban lethal autonomous weapons systems by the conclusion of the informal meeting . Some had noted that it helped form a “common understanding, but that questions still remained”(Report of the 2014). All of the above mentioned questions and concerns will be submitted to a formal meeting of the Conference of the Convention on Certain Conventional Weapons. Overall, lethal autonomous weapon systems is a highly debated topic. It is encouraging to know that it is being taken seriously and we will have to wait and see what the future brings.(Report of the 2014)






Works Cited
1, No. Framing Discussions on the Weaponization of Increasingly Autonomous Technologies (n.d.): n. pag. UNIDIR, 2014. Web. 17 Feb. 2015.
Hsu, Jeremy. "U.S. Navy Tests Robot Boat Swarm to Overwhelm Enemies." IEEE Spectrum. N.p., 5 Oct. 2014. Web. 17 Feb. 2015.
Kurzweil, Ray. "Don't Fear Artificial Intelligence." Time. Time, 19 Dec. 2014. Web. 02 Mar. 2015.
"Report of the 2014 Informal Meeting of Experts on Lethal Autonomous Weapons Systems (LAWS)." Meeting of the High Contracting Parties to the Convention on Prohibitions or Restrictions on the Use of Certain Conventional Weapons Which May Be Deemed to Be Excessively Injurious or to Have Indiscriminate Effects, 11 June 2014. Web. 01 Mar. 2015.
Schmitt, Michael N. FEATURES Autonomous Weapon Systems and International Humanitarian Law: A Reply to the Critics (n.d.): n. pag. 2013. Web. 17 Feb. 2015.
"UN Meeting Targets 'killer Robots'" UN News Center. UN, 14 May 2014. Web. 17 Feb. 2015.




http://time.com/3641921/dont-fear-artificial-intelligence/

http://www.unidir.org/files/publications/pdfs/framing-discussions-on-the-weaponization-of-increasingly-autonomous-technologies-en-606.pdf

https://ccdcoe.org/cycon/2013/proceedings/d2r1s9_sullins.pdf

http://harvardnsj.org/wp-content/uploads/2013/02/Schmitt-Autonomous-Weapon-Systems-and-IHL-Final.pdf

http://spectrum.ieee.org/automaton/robotics/military-robots/us-navy-robot-boat-swarm

http://www.un.org/apps/news/story.asp?NewsID=47794#.VO4ezXVGh5R 

http://techland.time.com/2012/11/20/should-we-ban-killer-robots-human-rights-group-thinks-so/

http://daccess-dds-ny.un.org/doc/UNDOC/GEN/G14/048/96/PDF/G1404896.pdf?OpenElement
